% This is a sample document using the University of Minnesota, Morris, Computer Science
% Senior Seminar modification of the ACM sig-alternate style. Much of this content is taken
% directly from the ACM sample document illustrating the use of the sig-alternate class. Certain
% parts that we never use have been removed to simplify the example, and a few additional
% components have been added.

% See https://github.com/UMM-CSci/Senior_seminar_templates for more info and to make
% suggestions and corrections.

\documentclass{sig-alternate}
\usepackage{color}
\usepackage[colorinlistoftodos]{todonotes}

%%%%% Uncomment the following line and comment out the previous one
%%%%% to remove all comments
%%%%% NOTE: comments still occupy a line even if invisible;
%%%%% Don't write them as a separate paragraph
%\newcommand{\mycomment}[1]{}

\begin{document}

% --- Author Metadata here ---
%%% REMEMBER TO CHANGE THE SEMESTER AND YEAR AS NEEDED
\conferenceinfo{UMM CSci Senior Seminar Conference, December 2015}{Morris, MN}

\title{UMM CSci Senior Seminar LaTeX template}

\numberofauthors{1}

\author{
% The command \alignauthor (no curly braces needed) should
% precede each author name, affiliation/snail-mail address and
% e-mail address. Additionally, tag each line of
% affiliation/address with \affaddr, and tag the
% e-mail address with \email.
\alignauthor
Chris S. Student\\
	\affaddr{Division of Science and Mathematics}\\
	\affaddr{University of Minnesota, Morris}\\
	\affaddr{Morris, Minnesota, USA 56267}\\
	\email{cssxxxx00000@morris.umn.edu}
}

\maketitle
\begin{abstract}
This paper provides a sample of a \LaTeX\ document which conforms,
somewhat loosely, to the formatting guidelines for the University of
Minnesota, Morris, Computer Science Senior Seminar proceedings.
It is based heavily on (and takes material directly from) a similar document 
illustrating the format of the ACM SIG Proceedings, which we have based our
proceedings format on.

The original ACM document tried to include
\begin{quote}
every imaginable sort
of ``bells and whistles", such as a subtitle, footnotes on
title, subtitle and authors, as well as in the text, and
every optional component (e.g. Acknowledgments, Additional
Authors, Appendices), not to mention examples of
equations, theorems, tables and figures.
\end{quote}

We've removed many of the more esoteric tricks here because either
they'll never be used (e.g., multiple authors) or are used \emph{very}
rarely (e.g., appendices). Refer to the original ACM document for more
of those fancy examples.
% The current paper format *only* allows inline comments using the todo
% macro. That's kind of a bummer, and it would be neat if someone figured
% out how to change the acmconf style to allow this. I suspect it isn't *hard*
% but there are quite a few details that have to be sorted out in synchrony.
\todo[inline]{Needs more work}
\end{abstract}

\keywords{ACM proceedings, \LaTeX, text tagging}

\section{Introduction}
\label{sec:introduction}


\section{Background Information}
\label{sec:body}


\subsection{The Memristor}
\label{sec:typeChangesSpecialChars}


\subsection{Crossbar Array}
\label{sec:mathEquations}


\subsubsection{Inline (In-text) Equations}
\label{sec:inlineEquations}


\subsubsection{Display Equations}
\label{sec:displayEquations}



\section{Memristor Based Architectures}



\subsection{CIM - Computation in Memory}
\label{sec:citations}



\subsection{Theorem-like Constructs}
\label{sec:theoremLikeConstructs}

Other common constructs that may occur in your article are
the forms for logical constructs like theorems, axioms,
corollaries and proofs.  There are
two forms, one produced by the
command \texttt{\textbackslash newtheorem} and the
other by the command \texttt{\textbackslash newdef}; perhaps
the clearest and easiest way to distinguish them is
to compare the two in the output of this sample document:

Theorem~\ref{thm:integration} below uses the \textbf{theorem} environment, created by
the \texttt{\textbackslash newtheorem} command:

% You would usually put a \newtheorem command up at the top 
% of your LaTeX document after the \usepackage commands. It's
% just here in this example so it's with the text that describes it.
\newtheorem{theorem}{Theorem}

\begin{theorem}
Let $f$ be continuous on $[a,b]$.  If $G$ is
an antiderivative for $f$ on $[a,b]$, then
\begin{displaymath}\int^b_af(t)dt = G(b) - G(a).\end{displaymath}
\label{thm:integration}
\end{theorem}

The other uses the \textbf{definition} environment, created
by the \texttt{\textbackslash newdef} command:
\newdef{definition}{Definition}
\begin{definition}
If $z$ is irrational, then by $e^z$ we mean the
unique number which has
logarithm $z$: \begin{displaymath}{\log e^z = z}\end{displaymath}
\end{definition}

Two lists of constructs that use one of these
forms is given in the
\textit{Author's  Guidelines}.
 
There is one other similar construct environment, which is
already set up
for you; i.e. you must \textit{not} use
a \texttt{\textbackslash newdef} command to
create it: the \textbf{proof} environment.  Here
is a example of its use:
\begin{proof}
Suppose on the contrary there exists a real number $L$ such that
\begin{displaymath}
\lim_{x\rightarrow\infty} \frac{f(x)}{g(x)} = L.
\end{displaymath}
Then
\begin{align*}
l &= \lim_{x\rightarrow c} f(x) \\
  &= \lim_{x\rightarrow c}
\left[ g{x} \cdot \frac{f(x)}{g(x)} \right ] \\
  &= \lim_{x\rightarrow c} g(x) \cdot \lim_{x\rightarrow c}
\frac{f(x)}{g(x)}  \\
  &= 0\cdot L  \\
  &= 0,
\end{align*}
which contradicts our assumption that $l\neq 0$.
\end{proof}

Complete rules about using these environments and using the
two different creation commands are in the
\textit{Author's Guide}; please consult it for more
detailed instructions.  If you need to use another construct,
not listed therein, which you want to have the same
formatting as the Theorem
or the Definition~\cite{salas:calculus} shown above,
use the \texttt{\textbackslash newtheorem} or the
\texttt{\textbackslash newdef} command,
respectively, to create it.

\subsection*{A {\secit Caveat} for the \TeX\ Expert}
\label{sec:caveatForExperts}

Because you have just been given permission to
use the \texttt{\textbackslash newdef} command to create a
new form, you might think you can
use \TeX's \texttt{\textbackslash def} to create a
new command: \textit{Please refrain from doing this!}
Remember that your \LaTeX\ source code is primarily intended
to create camera-ready copy, but may be converted
to other forms -- e.g. HTML. If you inadvertently omit
some or all of the \texttt{\textbackslash def}s recompilation will
be, to say the least, problematic.

\section{Conclusions}
\label{sec:conclusions}

This paragraph will end the body of this sample document.
Remember that you might still have Acknowledgments or
Appendices; brief samples of these
follow.  There is still the Bibliography to deal with; and
we will make a disclaimer about that here: with the exception
of the reference to the \LaTeX\ book, the citations in
this paper are to articles which have nothing to
do with the present subject and are used as
examples only.

\section*{Acknowledgments}
\label{sec:acknowledgments}

This section is optional; it is a location for you
to acknowledge grants, funding, editing assistance and
what have you.

You want to use the \texttt{\textbackslash section*} version of the \texttt{section}
command, as an acknowledgments section typically does \emph{not} get
a number.

It is common (but by no means necessary) for students to thank
their advisor, and possibly other faculty, friends, and family who provided
useful feedback on the paper as it was being written.

In the present case, for example, the
authors would like to thank Gerald Murray of ACM for
his help in codifying this \textit{Author's Guide}
and the \textbf{.cls} and \textbf{.tex} files that it describes.

% The following two commands are all you need in the
% initial runs of your .tex file to
% produce the bibliography for the citations in your paper.
\bibliographystyle{abbrv}
% sample_paper.bib is the name of the BibTex file containing the
% bibliography entries. Note that you *don't* include the .bib ending here.
\bibliography{sample_paper}  
% You must have a proper ".bib" file
%  and remember to run:
% latex bibtex latex latex
% to resolve all references

\end{document}
